\documentclass[11pt]{article}
\usepackage{fullpage}
\usepackage{setspace}
\usepackage{graphicx}
\usepackage{hyperref}
\usepackage{listings}
\usepackage{color}

\definecolor{dkgreen}{rgb}{0,0.6,0}
\definecolor{gray}{rgb}{0.5,0.5,0.5}
\definecolor{mauve}{rgb}{0.58,0,0.82}

\lstloadlanguages{Matlab} %use listings with Matlab for Pseudocode
\lstnewenvironment{PseudoCode}[1][]
{\lstset{language=Matlab,basicstyle=\scriptsize, keywordstyle=\color{darkblue},numbers=left,xleftmargin=.04\textwidth,#1}}
{}


\title{%
  Randomized Algorithm for shape pattern counting on GraphX \\
  \large independent project proposal}
\author{Supervisor: Prof.Ke YI, Student: GENG XU(20405672)}

\begin{document}

\maketitle

\vspace*{-0.8cm}

\section*{Introduction and Scope}

\item Counting the number of specific shapes is a fundamental problem in graph analysis. Intuitively, such an algorithm has a time complexity of $O(n^{m})$ where m is the number of edges in the shape. This project aims to solve the problem by two approaches:

\begin{itemize}
\item Instead of exact counting, a Monte Carlo algorithm based on random sampling is designed to provide a theoretically reliable solution.
\item GraphX takes the advantage of Spark and provide convenient programming interfaces for deploying such an algorithm on computing clusters.
\end{itemize}
\item This project will implement such an algorithm on GraphX (scala-Spark). Experiment and analysis will be carried out for coming up with performance measurements, including accuracy and speed up ratio.

\section*{Algorithm Design}
\textbf{input:} graph, number of edges in pattern, sample rate \\
\textbf{output:} number of pattern 

\begin{PseudoCode}
randCount(graph, n, rate):
	vertexSet<-graph.vertices.sample(rate)
	degreeGraph<-graph.mapEdges(e<-src.degree)
	for(1 to n):
		vertexSet.foreach{v=>nextStep=random pick 1 neighbor and validate}
		vertexSet.aggregateMsg(only if triplet.dstId==nextStep			
			p=p*1/edge.attr
			sendMsg: (pathStartId,path,p,#ofiteration,nextStep),
			rcvMsg: (a,b)=>a++b
		)
		updateGraph
	for end
	graph.vertices.filter(v=>v.vertexId==msg.pathStartId).sum(1/p)
end

validate:
	!path.contains(picked) || (#ofiteration==n && picked==pathStartId)
	
\end{PseudoCode}


\section*{Measurements and Experiment}

\item Some experiments are designed to evaluate the performance of the algorithm:\\


\begin{tabular}{|c|c|c|c|c|}
\hline
\hline
Group No. & algorithm & Mode & sample rate & Measurement\\
\hline
\hline
1 & Exact counting & cluster & 1 & time, counts  \\
\hline
2 & Random counting & local & 1 & time  \\
\hline
3 & Random counting & cluster & 1 & time, counts  \\
\hline
4 & Random counting & cluster & varying & time, counts  \\
\hline
\end{tabular}

\begin{itemize}
\item Speed up ratio due to randomization: Group 1 vs Group 3
\item Speed up ratio due to parallel computing: Group 2 vs Group 3
\item sample rate - speed up/accuracy: Group 1 vs Group 3,4
\end{itemize}

\section*{Deliverables}
\item This is a 3-credit bearing project supervised by Prof. Ke YI. The final deliverable will be runnable Spark jar and report for performance details.\\

\item Code repository: https://github.com/PhilipGeng/RandCounter\\

\section*{Reference}
\item F. Li, B. Wu, K. Yi, and Z. Zhao, “Wander Join,” Proceedings of the 2016 International Conference on Management of Data - SIGMOD '16, 2016. \\

\item R. S. Xin, J. E. Gonzalez, M. J. Franklin, and I. Stoica, “GraphX,” First International Workshop on Graph Data Management Experiences and Systems - GRADES '13, 2013.\\



\end{document}





